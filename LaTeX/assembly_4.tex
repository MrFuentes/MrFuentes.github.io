% !TEX root = \path\to\root\file.tex
\documentclass{article}
\title{Memory Mapping}

\begin{document}
\maketitle

\begin{center}
  \rule{0.5\textwidth}{0.4pt}
\end{center}

\section{Memory Mapping}
\begin{itemize}
  \item{The CPU has an \textbf{\textit{Address Space}}}
  \item{This \textbf{\textit{Address Space}} is the range of addresses the CPU can work with}
  \item{The \textbf{\textit{Address Space}} is defined by the number of lines in the address bus}
  \item{Some of our addresses are used for dealing with memory mapped devices}
  \begin{itemize}
    \item{This means the devices is linked to a specific memory location}
  \end{itemize}
  \item{Advantages of memory mapping:}
  \begin{itemize}
    \item{Very easy to program by treating the devices as a memory location}
  \end{itemize}
  \item{Disadvantages of memory mapping:}
  \begin{itemize}
    \item{You lose certain memory locations the the IO devices}
  \end{itemize}
\end{itemize}

\begin{center}
  \rule{0.5\textwidth}{0.4pt}
\end{center}

\section{The VDU}
\begin{itemize}
  \item{SMS32's VDU is a memory mapped device}
  \item{It is technically \textbf{\textit{Write-Only}}}
  \item{It uses the address space from \textit{C0}-\textit{FF}}
  \item{It accepts hexadecimal bytes}
  \item{It displays the ASCII equivalent character}
  \item{the screen is 4 rows of 16 characters}
\end{itemize}

\begin{center}
  \rule{0.5\textwidth}{0.4pt}
\end{center}

\end{document}
