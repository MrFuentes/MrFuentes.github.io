\documentclass{article}
\title{Jumps}

\begin{document}
\maketitle

\begin{center}
  \rule{0.5\textwidth}{0.4pt}
\end{center}

\section{Jumping}
\begin{itemize}
  \item{Normally, the PC register tells the CPU where the next instruction is}
  \item{If you want your CPU to \textit{jump} to another place in the program, you use the \textbf{\textit{JMP}} instruction}
  \item{For Example:}
  \begin{itemize}
    \item{You might want to write \textit{JMP 40}, to jump to memory location 40}
    \item{However in SMS32, you cannot do this}
    \item{If you want to jump to another section of code you must:}
    \begin{enumerate}
      \item{Identify that place with a \textit{label}}
      \item{Write \textit{JMP label}}
    \end{enumerate}
  \end{itemize}
  \item{Jumps can go either direction:}
  \begin{itemize}
    \item{Back up through the program}
    \item{Further on in the program}
  \end{itemize}
\end{itemize}

\begin{center}
  \rule{0.5\textwidth}{0.4pt}
\end{center}
\pagebreak

\section{Conditional Jumping}
\begin{itemize}
  \item{Rather than using the \textbf{\textit{JMP}} instruction which will jump the specified label no matter what, you can use a \textit{Conditinal Jump}}
  \begin{enumerate}
    \item{\textbf{\textit{JZ}}}
    \begin{itemize}
      \item{Only Jump if the \textit{Zero Flag} is set}
    \end{itemize}
    \item{\textbf{\textit{JNZ}}}
    \begin{itemize}
      \item{Only Jump if the \textit{Zero Flag} is \underline{Not} set}
    \end{itemize}
    \item{\textbf{\textit{JS}}}
    \begin{itemize}
      \item{Only Jump if the \textit{Sign Flag} is set}
    \end{itemize}
    \item{\textbf{\textit{JNS}}}
    \begin{itemize}
      \item{Only Jump if the \textit{Sign Flag} is \underline{Not} set}
    \end{itemize}
    \item{\textbf{\textit{JC}}}
    \begin{itemize}
      \item{Only Jump if the \textit{Carry Flag} is set}
    \end{itemize}
    \item{\textbf{\textit{JNC}}}
    \begin{itemize}
      \item{Only Jump if the \textit{Carry Flag} is \underline{Not} set}
    \end{itemize}
    \item{\textbf{\textit{JI}}}
    \begin{itemize}
      \item{Only Jump if the \textit{Interrupt Flag} is set}
    \end{itemize}
    \item{\textbf{\textit{JNI}}}
    \begin{itemize}
      \item{Only Jump if the \textit{Interrupt Flag} is \underline{Not} set}
    \end{itemize}
  \end{enumerate}
\end{itemize}

\begin{center}
  \rule{0.5\textwidth}{0.4pt}
\end{center}

\section{Relative Jumps}
\begin{itemize}
  \item{The main drawback to using jumps is that you can only jump so far from the IP}
  \item{Relative jumps mean that we don't say what memory location we jump to}
  \item{Instead we say how far we want to jump from the current instruction}
  \item{This is done using the \textbf{\textit{ORG}} instruction after using the \textbf{\textit{JMP}} instruction}
\end{itemize}

\end{document}
